

\begin{abstract}%
%
%\PARstart{T}{he} 
%

%% Problem statement 
\Gls{BNN} models provide an important means to improve understanding of neural processing of sound.
The quality of BNN models is dependent upon the experimental data and optimisation methods used in their development.
The \CNSM provides a robust
spectral representation of sound and plays an essential part in speech
communication.
Existing \BNN models of the \CNSM have not been sufficiently considerate of the most current
physiological data.  Additionally, optimisation methods used in these models have not been adequately documented.


%% This is what I did to address the issue
This thesis presents a novel \BNN model of the \CNSM.  The model was optimised using rigorous sequential methods and
simultaneous \GAs. 

%% Study 1 did:
The first investigation developed the neural models and synaptic connections of
the \CNSM model in a sequential optimisation procedure.  \Gls{ANF} input to the CNSM
model used phenomenologically-accurate output of the Carney AN model.  Each
\CNSM cell type model was designed using appropriately detailed neural models. Their
synaptic parameters were optimised based on established
physiological data.


%% Study 2 did
The second investigation analysed the output responses of the optimised \CNSM
model to \AM tones.  Encoding of \AM tones undergoes important transformations in
the \CN from purely spike-timing temporal encoding in \ANFs to feature-based
rate encoding in higher auditory centres. Rate and temporal information was
analysed in each CNSM cell type in response to changes in sound level and
modulation frequencies.  The \CNSM model adequately reproduced experimental
responses.


%% Study 3 did
The third investigation used \GAs to simultaneously fit all network parameters to
a simplified CNSM model.  Pre-defined target parameters were used to generate
surrogate data for the optimisation to fit.  The ability of novel \BNN cost
functions to constrain the model was also investigated,  including (1) dynamic spike-time programming
(finding minimum distance between two spike trains), (2) instantaneous firing
rate comparisons, and (3) average intracellular voltage comparisons.
 

%% Summary of thesis contribution
 
Through the design and optimisation of the CNSM model, the thesis demonstrates the utility of 
evidence-based sequential methods and whole-network simultaneous optimisation methods using \GAs for \BNN development.
 



\end{abstract}

\clearpage
